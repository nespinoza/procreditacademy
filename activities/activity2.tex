\documentclass{tufte-handout}

%\geometry{showframe}% for debugging purposes -- displays the margins
\usepackage{amsmath}
\setcitestyle{authoryear,open={(},close={)}}
% Set up the images/graphics package
\usepackage{graphicx}
\setkeys{Gin}{width=\linewidth,totalheight=\textheight,keepaspectratio}
\graphicspath{{graphics/}}

\title{Activity 2: The Period-Luminosity Relation}
\author[Dr. N\'estor Espinoza]{Dr. N\'estor Espinoza}
\date{January, 2019}  % if the \date{} command is left out, the current date will be used

% The following package makes prettier tables.  We're all about the bling!
\usepackage{booktabs}

% The units package provides nice, non-stacked fractions and better spacing
% for units.
\usepackage{units}

% The fancyvrb package lets us customize the formatting of verbatim
% environments.  We use a slightly smaller font.
\usepackage{fancyvrb}
\fvset{fontsize=\normalsize}

% Small sections of multiple columns
\usepackage{multicol}

% Provides paragraphs of dummy text
\usepackage{lipsum}

% These commands are used to pretty-print LaTeX commands
\newcommand{\doccmd}[1]{\texttt{\textbackslash#1}}% command name -- adds backslash automatically
\newcommand{\docopt}[1]{\ensuremath{\langle}\textrm{\textit{#1}}\ensuremath{\rangle}}% optional command argument
\newcommand{\docarg}[1]{\textrm{\textit{#1}}}% (required) command argument
\newenvironment{docspec}{\begin{quote}\noindent}{\end{quote}}% command specification environment
\newcommand{\docenv}[1]{\textsf{#1}}% environment name
\newcommand{\docpkg}[1]{\texttt{#1}}% package name
\newcommand{\doccls}[1]{\texttt{#1}}% document class name
\newcommand{\docclsopt}[1]{\texttt{#1}}% document class option name


\titlespacing*{\section}{0pt}{0.3\baselineskip}{\baselineskip}
\begin{document}

\noindent\textcolor{Red}{\rule{16cm}{3mm}}

{\let\newpage\relax\maketitle}
%\maketitle% this prints the handout title, author, and date

\begin{abstract}
\noindent\textcolor{Red}{\rule{10cm}{0.4mm}}
\noindent In this activity, we will collaboratively measure the distance to the 
galaxy M100 (NGC 4321) using known Cepheid variables that \textit{you} will have to find, 
along with the period-luminosity relation described by Henrietta Lewit in 1912.

\noindent\textcolor{Red}{\rule{10cm}{0.4mm}}
\end{abstract}


\section{Instructions}\label{sec:intro}
\begin{fullwidth}
Using your computer and/or your cellphone, go to the following website: \url{http://astro.wku.edu/labs/m100/thehunt.html} 
(this website has been put up by Diane Dutkevitch and the The Northwestern University Astronomy Web Lab Series). With this 
at hand, follow the following instructions:

\begin{enumerate}
\item Choose one of the grids in the webpage, and search for its Cepheid variable. Once you find it, write down the period of 
the variable and its \textit{apparent magnitude}, $m_v$.
\item Use the Leavitt's period-luminosity relation to find the \textit{absolute magnitude}, $M_V$, of the star. The relation is 
$M_V = -[2.76(\log_{10}P - 1.0)] - 4.16$.
\item Go and write in the whiteboard the distance modulus, $D = m_V-M_V$.
\item Using the distance modulus from all of your classmates, compute the distance to M100 by using the formulae $d = 10^{0.2(\bar{D}+5-A_V)}$, where $A_V = 0.25$ and $\bar{D}$ is the average distance modulus obtained for all the Cepheids.
\end{enumerate}

\end{fullwidth}

%\begin{enumerate}
%\item Introduction (day 1).
%\item Cosmology, pt. I (day 2).
%\item Cosmology, pt. II (day 3).
%\item Stellar formation and evolution (day 4).
%\item Exoplanets (day 5).
%\end{enumerate}
%We describe each of these units below.
%\end{fullwidth}
%\subsection{Headings}\label{sec:headings}

%\begin{table}[ht]
%  \centering
%  \fontfamily{ppl}\selectfont
%  \begin{tabular}{ll}
%    \toprule
%    Margin & Length \\
%    \midrule
%    Paper width & \unit[8\nicefrac{1}{2}]{inches} \\
%    Paper height & \unit[11]{inches} \\
%    Textblock width & \unit[6\nicefrac{1}{2}]{inches} \\
%    Textblock/sidenote gutter & \unit[\nicefrac{3}{8}]{inches} \\
%    Sidenote width & \unit[2]{inches} \\
%    \bottomrule
%  \end{tabular}
%  \caption{Here are the dimensions of the various margins used in the Tufte-handout class.}
%  \label{tab:normaltab}
%  %\zsavepos{pos:normaltab}
%\end{table}


\end{document}
